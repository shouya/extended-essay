%%% Local Variables:
%%% mode: latex
%%% TeX-master: "ee"
%%% End:

\section{Specification}
\subsection{Functional Programming Paradigm}
In functional programming, a function is a first-class
member. A function treated as a regular data type, no more special
than an integer or a string. It is both data and code, as it's
executable.

To program in functions gives me a feeling like to translate complex
programming procedures into mathematical patterns. The steps and
manner to abstract a procedure into a function was marvelous.
To keep a function pure, out of interference by external
environment, to get rid of states and mutable variables, programmer
should focus on the function's application. The things you need to
consider is the what the function is and how it is defined, just in
mathematical language, in contrast to imperative programming paradigm,
in which you need to think of translating an obvious declaration into
intelligible statements.

Due to its simplicity and flexibility,

\subsection{Lisp Specification}
Lisp is an old programming language which a bunch of revolutionary
features. Such as garbage collection, dynamic typing,
\subsection{Scheme Specification}



