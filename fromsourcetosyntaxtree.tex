\section{From Source Code to Syntax Tree}
The process of forming a abstract syntax tree from source code is
usually done by two part of works. First the source code will be
scanned and split into small tokens with singular meaning. This part
of job is done by an advanced string parser, which is what we called
lexical scanner. After proceeded by a lexical scanner, or lexer for
simple, the code will be in a number of linear tokens. Then we form these
tokens into an abstract syntax tree. By following the syntax of a
programming language, a source code could have its meaning. For example,
consider the following operation code: \verb 1+1 . The lexical scanner will turn
it into token list: an integer \verb 1 , a symbol \verb + ,
and another integer \verb 1 . Then the syntax parser will recognize
the this code into an binary plus operation, it takes two integer
\verb 1  as its operands.


\subsection{Lexical Scanner}
% \sourceofthissectionbasic{


The lexical scanner is written completely by myself. Firstly LISP has
very few types of token that is quite simple so it might be not necessary to
use a lexical scanner generator like
lex\footnote{lex: \url{https://en.wikipedia.org/wiki/Lex_(software)}, or its
  successor
  \href{https://www.gnu.org/software/flex/}{lex}. The
  ruby port of lex \href{https://rubygems.org/gems/rex}{rex}.}. I
wish I can hold more controls on the scanner since I will implement a
REPL\footnote{Read-evaluate-print loop, an interactive interface for
  code executing} and it requires to communicate with the lexical
scanner.

My lexical scanner is implemented as a class. An instantiated scanner
can scan source code from standard input, string, file, or any IO
stream.

For most basic usage, this scanner can be regarded as an enumerable
object, so you can just specify the input and then fetch an
array of tokens directly from
it. (\href{\libcoderefln{scanner.rb}{154}}{usage
  example})

The principle of this lexical scanner is simple. It load the source
code string into memory, and set the position pointer to the head of
the string. Then it traverse the lexical rule list until it find one
that matches the string from current position. Afterward it do action
specified by that rule, and move the position pointer to skip the
matched text. The scanner loops then, until the it meets EOF(end-of-file).

A technical problems need to be noticed is the \emph{state}. A scanner could
behave differently when it meets the same text under different
state. This method is necessary to cope with complex syntax patterns.
For example, the word \verb+if+ means differently in a direct context,
string context, or a comment context. What the context usually
deliminators do is transfer state. (code example:
\libcodehrefln{scanner_rules.rb}{32}{string context},
\libcodehrefln{scanner_rules.rb}{10}{comment context})

Each token the scanner parsed produce a pair in format of $[type,
value]$, and for the last token, EOF, it returns $[false,
false]$. This agreement is for the syntax parser to recognize what a
token is, and specified by racc.

\subsection{Syntax Parser}
% \sourceofthissectiona{lib/revo/grammar.y}
The syntax parser is to form an abstract syntax tree from a linear token
composition produced by a lexical scanner. The most popular syntax
parser is
\href{http://dinosaur.compilertools.net/yacc/}{YACC}\footnote{YACC: Yet
  Another Compiler Compiler}, or it's GNU successor,
\href{https://www.gnu.org/software/bison/}{bison}. The ruby port of
YACC is called \href{https://github.com/tenderlove/racc}{racc}. This
is what I adopt as the parser generator for \revo.

racc has a DSL(domain specific language) for the syntax rule
file. Similar to YACC, a syntax rule can be defined in regular
BNF(\href{https://en.wikipedia.org/wiki/Backus-Naur_form}{Backus–Naur
  Form}). With such a powerful parser generator, all I need to do is
specify the various reduction rules for LISP. LISP has very simple
syntax, so the syntax specification is short as well. Mainly it
consists just a group of rules for LISP atoms, which includes basic
data types and identifiers, and another for the
list syntax. An atom or a list is reduced into a LISP expression.

Everything will be finally reduced to a LISP expression. Strictly, we
don't consider a bunch of LISP expressions, or no expression, as a single
expression to be returned by the syntax parser. In a regular scheme
source file, it is normal to have more than two expressions. The way
to solve this problem is the \verb+begin+ operator. The \verb+begin+
operator in Scheme takes any number of expressions as argument,
execute them and return the value of the last one. That part of rules
are illustrated below(\libcodehref{grammar.y}{16}{source position}):
\begin{verbatim}
         main: multi_expr        { val[0] }
             | expr              { val[0] }
             | /* empty */       { NULL }

   multi_expr: multi_expr_x      { Cons[:begin, val[0]] }

 multi_expr_x: expr expr         { Cons[val[0], Cons[val[1], NULL]] }
             | expr multi_expr_x { Cons[val[0], val[1]] }
\end{verbatim}

So far the code is formed well into a syntax tree. What I do in this
section is like to build a body for the plain text source. The next
step, the runtime, is a kind of magical --- to make it come alive.
