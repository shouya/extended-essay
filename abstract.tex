\section*{Abstract}

\begin{abstract}

This essay is about the research and study of scheme
programming language interpreter implementing.

Functional programming paradigm is one of the most mathematically
beautiful things I met in computer science. Scheme is one of the most
pure programming languages. I like this two things always. Thus I
tried to learn more about functional programming and implemented a
scheme interpreter by myself.

In this essay, I will bring up what I have done and what I have learnt
from this project. Practically I tried two times on
implementing this scheme interpreter. I'll separate a chapter to
discuss how the first trials failed and the improvement brought by the
current one.

This interpreter is named \revo. The word '\revo' means 'dream' in
\href{https://en.wikipedia.org/wiki/Esperanto}{Esperanto}. Originally
I was planned to implement an operating system names \fantazio, which
means 'fantasy', with \revo~as it's primitive interpreter. But as far
\fantazio~project has not been started.


\begin{comment}
At the first trial I boxed all
data types into ruby classes, this caused a low extensibility and
efficiency of implementing the library. That is a very basic version,
which has the very beginning features and unhygienic macro, which is
much simpler to implement than the hygienic one. Contrarily, the new
one adopt the some structures from the
\href{https://github.com/jcoglan/heist}{heist project}, which is
another implementation of scheme in ruby. Heist uses native ruby ones
as the scheme's primitive types, which is what I followed in the new
implementation.
\end{comment}


The codes of the project is open-sourced under MIT license. You may get
them from
\hyperlink{github/shouya/revo}{https://github.com/shouya/revo},
Firstly of all, you could see the ascii-cast of a demonstration of
the interpreter I recorded months ago to preview what you can do in
this interpreter.\footnote{This demonstration is just for the REPL,
  the main interpreter is to execute source file}

\end{abstract}