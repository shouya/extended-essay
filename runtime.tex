\section{Runtime}
The usage of runtime is to create a space for the code to
execute. Basically a runtime provides a scope stack for the local
variables to store, and trace the call stack for the function
invoking.

\subsection{Scope Model}
Scopes in \revo\ is only for storing local symbols, which is generally
known as variables. Technically, these symbols are stored in a hash
table, and is referable with their names.

\subsubsection{Free Variables}
Usually, a scope has a parent, this is what we called 'scope nesting'.
When the program requested a variable in a scope, it will firstly
search its own symbol table to see if the variable is defined
locally. If that variable is not found, it will ask its parent for
it. It does this recursively until the variable is found, or it will
raise a '\emph{name-not-found}' exception and interrupt the execution.

A creation of a scope is done by closure evaluation. A closure is
usually known as a lambda($\lambda$) in functional
programming. Concisely a closure could be regarded as a composition of
a function, which is a bunch of codes with optional parameters to accept input and
returns a value, and a scope pointer. A lambda will reserve the
pointer to the scope where it is created, and when it evaluates, a new
scope will be created and its parent will be set as the one it
reserves.

So consider this code snippet:
\begin{verbatim}
(define my-lambda       ;; define the lambda with the name 'my-lambda'
  (let ((x 1))          ;; start a local scope and set x to 1
    (lambda () (x))))   ;; create a lambda that return the value of x
(define x 2)            ;; define a global variable x, set it to 2

(display (my-lambda))   ;; it shows 1 instead of 2
\end{verbatim}

Because \verb+my-lambda+ is created under the scope where \verb+x+ is
set to \verb+1+, it will regard this as the parent to search for
variables, no matter what scope it is evaluated at.

The situation I mentioned above is the way how \revo\ deal with
free variables. A free variable, as its name, it not bounded on any
specified value. The concept might be a little complicated to
understand, but I'll explain with a simple example:
\begin{verbatim}
(lambda (a b) (+ a b c))
\end{verbatim}

In this example, it is a anonymous lambda with two parameters \verb+a+
and \verb+b+. When we invoke this lambda, we have to specify two
arguments for it, let's say, \verb+1+ and \verb+2+. In this case,
\verb+1+ will be bounded on \verb+a+, and \verb+2+ will be bounded on
\verb+b+. For the name \verb+c+, the variable it represent is not
designated when the lambda is invoked, so this is what we called 'free
variable'.

\subsubsection{Bounded Variable}
I used a quick way to cope with bounded variables for lambdas. In the
new scope created by a called lambda, I will put a group of variables
with the name of its parameters and the value of the arguments passed
into the symbol table of the scope. So referring its arguments in the
body is right like referring local variables. Actually there's no
concept of local scope in Scheme. Usually, the operator \verb+let+ to
start a local scope is also done by lambda. In the standard template
syntax, \verb+let+ is defined like:

\begin{verbatim}
(define-syntax let
  (syntax-rules ()
    ((let ((name val) ...)
       body ...)
     ((lambda (name ...) body ...)
      val ...))))
\end{verbatim}

So as shown, \verb^(let ((a 41)) (+ a 1))^ is equivalent to
\verb^((lambda (a) (+ a 1)) 41)^, which is no different from calling a
temporary lambda.

